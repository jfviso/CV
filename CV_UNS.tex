\documentclass[11pt,a4paper,sans]{moderncv}   % opciones posibles incluyen tamaño de fuente ('10pt', '11pt' and '12pt'), tamaño de papel ('a4paper', 'letterpaper', 'a5paper', 'legalpaper', 'executivepaper' y 'landscape') y familia de fuentes ('sans' y 'roman')

% temas de moderncv
\moderncvstyle{casual}                        % las opciones de estilo son 'casual' (por omision),'classic', 'oldstyle' y 'banking'
\moderncvcolor{black}

% ajustes para los margenes de pagina
\usepackage[scale=0.85]{geometry}
\setlength{\hintscolumnwidth}{2.4cm}           % si desea cambiar el ancho de la columna para las fechas
\usepackage[spanish,english]{babel}
\usepackage[utf8]{inputenc}

% para que las secciones arranquen en pagina nueva si no entran.
\usepackage{needspace}

% datos personales
\name{Juan Francisco}{Viso}
\title{Curriculum Vitae}
\address{Thompson 271}{8000 Bahía Blanca - Argentina}
\phone[mobile]{+54-291-4261108}
\phone[fixed]{+54-291-4516111}
\email{jfviso@hotmail.com}

\usepackage{csquotes}
\usepackage{textgreek} %to use greek letters.

\def\FormatName#1{%
  \def\myname{Viso J.F.}%
  \edef\name{#1}%
  \ifx\name\myname
    \underline{#1}%
  \else
    #1%
  \fi}

%To compile use:
%pdflatex -synctex=1 -interaction=nonstopmode %.tex|bibtex courseschol.aux|bibtex papers.aux|bibtex posters.aux|bibtex courses.aux|bibtex other.aux|pdflatex -synctex=1 -interaction=nonstopmode %.tex|evince %.pdf
\usepackage[resetlabels]{multibib}
\newcites{courseschol,papers,posters,courses,other}{,,,,}
%----------------------------------------------------------------------------------
%            contenido
%----------------------------------------------------------------------------------
\begin{document}
\maketitle
\vspace{-1cm}

\needspace{5\baselineskip}
\section{Perfil Profesional}
\cvitem{}{Licenciado en Física, pronto a Doctorarse, con experiencia en modelado computacional, particularmente en Dinámica Molecular (Gromacs).
Experiencia en programación. Más de siete años de trabajo como docente universitario.
Personalidad proactiva y comunicativa con habilidad para adaptarse a los cambios y participar en equipos de trabajo de investigación y docencia}

\needspace{5\baselineskip}
\section{Formación Académica}
\cventry{2012--Presente}{Estudiante del Doctorado en Física}{Universidad Nacional del Sur}{Bahía Blanca}{}{Manuscrito entregado a los jurados.}
\cventry{2007--2012}{Licenciado en Física}{Universidad Nacional del Sur}{Bahía Blanca}{}{}
\cventry{2006}{Bachiller en Ciencias Naturales}{Colegio Del Solar}{Bahía Blanca}{}{}

\needspace{5\baselineskip}
\section{Idiomas Extranjeros}
\cvitemwithcomment{Ingles}{Nivel: First Certificate In English}{}
\cvitemwithcomment{Italiano}{Nivel: Livello Medio Terzo Anno}{}
\cvitemwithcomment{Alemán}{Nivel: Grundstufe II}{}

\needspace{5\baselineskip}
\section{Tesis de Doctorado}
\cvitem{Título}{\emph{Cálculo de Modos Normales en Cápsides Virales: aplicación en Triatoma Virus y comparación con otros virus icosaédricos.}}
\cvitem{Director}{Marcelo D. Costabel}
%\cvitem{descripci\'on}{Una breve descripci\'on de la tesis}

\needspace{5\baselineskip}
\section{Experiencia Docente}
\cventry{2015--Presente}{Asistente de Docencia}{Depto. de Física, Universidad Nacional del Sur}{}{}{Física II-IC (Para Ingeniería Civil - \textbf{Cargo ocupado actualmente}).\\ Mecánica del Continuo (Para Licenciatura en Física).\\ Física Aplicada T (Para Tecnicatura Universitaria en Operaciones Industriales).}
\cventry{2018}{Profesor}{Depto. de Física, Universidad Nacional del Sur}{}{}{Curso de Nivelación en Física.}
\cventry{2012--Presente}{Ayudante de Docencia A}{Depto. de Física, Universidad Nacional del Sur}{}{}{Física Arq (Para Arguitectura - \textbf{Cargo Ocupado Actualmente}).\\ Física General (Para Bioquímica - \textbf{Cargo Ocupado Actualmente}).\\Física Aplicada, Termodinámica y Laboratorio I (Para Licenciatura en Física).\\ Física I (Para Ingeniería Industrial).}
\cventry{2013-15-16-17}{Ayudante de Docencia A}{Depto. de Física, Universidad Nacional del Sur}{}{}{Curso de Nivelación en Física.}
\cventry{2010--2012}{Ayudante de Docencia B}{Depto. de Física, Universidad Nacional del Sur}{}{}{Mecánica (Para Licenciatura en Física).\\ Física I (Para Ingeniería Industrial).}

\needspace{5\baselineskip}
\section{Cursos de Posgrado Dictados}
\cventry{2016}{Biofísica de Sistemas Biológicos: Modelización de la Estructura, Dinámica, Estabilidad y Reacciones Químicas}{Depto. de Física, UNS}{}{}{A cargo del tutorial de Dinámica Molecular.}

\needspace{5\baselineskip}
\section{Aplicaciones a la Industria}
\cventry{2017}{Implementación de un modelo de optimización en lenguaje Python para la empresa Profertil}{}{}{}{}

\needspace{5\baselineskip}
\section{Gestión Académica}
\cventry{2015--2016}{Miembro del Consejo Departamental}{Depto. de Física, Universidad Nacional del Sur}{Representante del claustro de Auxiliares}{}{}
\cventry{2012}{Integrante del Colegio Electoral}{Depto. de Física, Universidad Nacional del Sur}{Representante del claustro de Auxiliares}{}{}
\cventry{2010--2012}{Miembro del Consejo Departamental}{Depto. de Física, Universidad Nacional del Sur}{Representante del claustro de Alumnos}{}{}

\needspace{5\baselineskip}
\section{Becas}
\cventry{2014}{Beca Interna Doctoral}{}{}{}{Beca otorgada por el CONICET para finalizar los estudios de Doctorado}
\cventry{2012}{Beca de Postgrado Tipo I}{}{}{}{Beca otorgada por el CONICET para inicializar los estudios de Doctorado}
\cventry{2011}{Beca de Introducción a la Investigación para Alumnos Avanzados}{}{}{}{Beca otorgada por la UNS para inicializar a los alumnos de grado en tareas de investigación}

\needspace{5\baselineskip}
\section{Becas para Asistir a Cursos}
\nocitecourseschol{*}
\bibliographystylecourseschol{plainyrrev}
\bibliographycourseschol{courseschol}

\needspace{5\baselineskip}
\section{Antecedentes de Investigación}
\cventry{2015--2018}{“Física Molecular de Biopolímeros (PARTE VI) Estudios de  estructura,  dinámica, modelización y plegamiento”}{PGI 24/F064}{Director: Marcelo D. Costabel}{Grupo de Biofísica, Depto. de Física, Universidad Nacional del Sur}{}
\cventry{2011--2014}{“Física Molecular de Biopolímeros (PARTE V) Estudios de  estructura,  dinámica, modelización y plegamiento”}{PGI 24/F025}{Director: Marcelo D. Costabel}{Grupo de Biofísica, Depto. de Física, Universidad Nacional del Sur}{}

\needspace{5\baselineskip}
\section{Estadías en Centros de Investigación}
\cventry{2014}{Estadía en el laboratorio del Dr. Diego Guérin en el Grupo de Cristalografía de Proteínas y Virus en la Unidad de Biofísica de la Universidad del País Vasco (UPV-EHU)}{Financiado por el programa de “Pasantías en Centros de Investigación Destinados a Jóvenes Docentes” de la UNS y por la Fundación de Biofísica Bizkaia}{}{}{}

\needspace{5\baselineskip}
\section{Publicaciones Científicas}
\nocitepapers{*}
\bibliographystylepapers{plainyrrev}
\bibliographypapers{papers}

\needspace{5\baselineskip}
\section{Presentación de Trabajos Científicos en Congresos}
\nociteposters{*}
\bibliographystyleposters{plainyrrev}
\bibliographyposters{posters}

\needspace{5\baselineskip}
\section{Cursos de Posgrado}
\nocitecourses{*}
\bibliographystylecourses{plainyrrev}
\bibliographycourses{courses}

\needspace{5\baselineskip}
\section{Actividades de Extensión Universitaria}
\cventry{2015}{Participación en el proyecto “Qué Onda con la Luz”}{Actividad de divulgación donde se propone despertar la curiosidad por fenómenos naturales donde la luz es protagonista y así poder explicar en términos simples la razón de su ocurrencia}{}{}{Avalado por el Depto. de Física de la UNS y el Instituto de Física del Sur.}
\cventry{2011--2014}{Participación en el proyecto “Curiosos en Acción”}{Este proyecto consiste en un Taller de Ciencias para docentes de nivel Inicial, Primario y Secundario}{}{}{Avalado por el Depto. de Física de la Universidad Nacional del Sur.}

\needspace{5\baselineskip}
\section{Lenguajes y Aplicativos Computacionales}
\cvitem{$\bullet$}{Experiencia en programación en lenguaje Python, Fortran, C y R.}
\cvitem{$\bullet$}{Experiencia en Dinámica Molecular (Gromacs) y técnicas de Muestreo Avanzado (Potential of Mean Forces y Umbrella Sampling).}
\cvitem{$\bullet$}{Experiencia en sistemas operativos tipo Unix.}
\cvitem{$\bullet$}{Experiencia en Shell Scripting en Bash y Python.}

\needspace{5\baselineskip}
\section{Otras Actividades}
\nociteother{*}
\bibliographystyleother{plainyrrev}
\bibliographyother{other}

\end{document}